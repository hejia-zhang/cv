\documentclass[a4paper, 12pt]{cv}
\geometry{left=2.0cm, top=1.5cm, right=2.0cm, bottom=2.0cm, footskip=.5cm}
\begin{document}
\name{Hejia Zhang}
\address{3335 S Figueroa St. Apt. 245}{Los Angeles}{CA 90007}
\contact{hejiazha@usc.edu}{(213) - 477 - 0490}
\section{EDUCATION}
\school{University of Southern California}{Los Angelas, USA}{M.S. in Computer Science (Intelligent Robotics)}{Jan. 2018--PRESENT}
\selectedcourses{Robotics}{Analysis of Algorithms}{}{}{}
\school{Zhejiang University}{Hangzhou, China}{B.E. in Bioengineering}{Sept. 2013--July. 2017}
\schoolhornors{Second-Class Scholarship for Outstanding Merits, Zhejiang University}{2013--2014}{Third-Class Scholarship for Outstanding Merits, Zhejiang University}{ 2015--2016}
\section{PROFESSIONAL EXPERIENCE}
\company{Shuneng Technology LLC}{Hangzhou, China}{Intern Data  Analyst}{July. 2016--Sept. 2016}
\workdescription{Assisted in data retrieval and movie box performance prediction for all movies for all cinemas in
China}{Assisted in analyzing data including media indices, box presales, types, casts of the movies and their
box performance using Artificial Neural Network}{Wrote a report about the results of box office performance forecasting}
\company{Seeta Technology Co., Ltd}{Beijing, China}{Softeware Engineer}{June. 2017--Dec. 2017}
\workdescription{Implemented a face access control system based on boost.asio, Poco Library and Face Recognition technology}{Improved and matained the functionality, stability and availabity of the face recognition platform which provides the basic face detection, face recognition and feature management api and is used in every project in company}{Participated in  developing a data annotation platform in Python with Tornado}
\section{ACADEMIC EXPERIENCE}
\project{Undergraduate Research, Institute of Biosystem Automation and Information Technology (Prof. Hui Fang)}{Zhejiang University, Hangzhou, China}{Researcher for "Desktop Application for Processing Plant’s Point Cloud Data"}{Feb. 2016--June. 2017}
\projectdescription{Responsible for the Graphical User Interface and several function modules of the application like leaf area measurement, plant’s
height measurement}{Developed the Graphical User Interface in C++ with Qt and the processing function modules with PCL}{Provided the application to producers and scientists for processing the plant’s point cloud data with ease and speed}
\project{2016 ASABE Robotics Student Design Competition}{Orlando, USA}{Designer \& Programmer}{March. 2016--July. 2016}
\projectdescription{Responsible for the design and programming of a fully automated robotic system that can simulate the transfer of fruits from the
harvester to the processing plant}{Developed the control system on Arduino Mega 2560 and realized the vision system in C++ with OpenCV on Raspberry 3 Model B}{ Designed a manipulator that can grab the target and suck the target into a storage compartment inside the robot}
\project{2015 National Intelligent Agricultural Equipment Innovation Competition for
College Students}{Jiangsu, China}{Designer \& Programmer}{Oct. 2015--Dec. 2015}
\projectdescription{Responsible for the design and programming of a fully automated robotic system that can simulate the tractor’s traversing in field which can avoid obstacles automatically}{Developed the path planning algorithm on Arduino Mega 2560 combined with information from infra-red sensors and ultrasonic sensors}{Won the 2nd prize in the competition}
\project{Undergraduate Research Internship, Bhalerao Lab (Prof. Kaustubh Bhalerao, Dr. Abhishek S. Dhoble)}{University of Illinois at
Urbana-Champaign, USA}{Researcher for "Modeling Microbial Deversity of Anaerobic Digestion through STELLA"}{July. 2015--Aug. 2015}
\projectdescription{Responsible for a methodology to account for microbial diversity in complex but structured models and the resulting model remains
powerful in representing macroscopic experimental data, but is moreover able to get insight in underlying microscopy}{Adaptation of microorganisms to perturbations and inhibitory substances, as suggested in this model, can significantly improve
anaerobic digestion process and thereby wastewater treatment efficiency}{Constructed the visual mathematical model with tool of STELLA}
\section{SKILLS}
\skill{Programming Languages:}{C/C++}{Python}{Matlab}{Shell Scripting}{}
\skill{Network Programming Libraries:}{Poco (C++)}{Tornado (Python)}{Mosquitto}{Boost.Asio}{}
\skill{Computer Vision Programming Libraries:}{OpenCV}{Point Cloud Library (PCL)}{}{}{}
\skill{Machine Learning Libraries:}{TensorFlow}{Scikit-Learn}{}{}{}
\skill{Robotics Development Framework:}{Robot Operating System (ROS)}{}{}{}{}
\section{REFERENCES}
\referee{Dr. Hui Fang  \bf{Associate Professor of Zhejiang University}  \normalfont{China}}{Dr. Huanyu Jiang  \bf{Professor of Zhejiang University}  \normalfont{China}}{Dr. John Zhang  \bf{President of Systems Analytics Inc.,}  \normalfont{USA}}
\end{document}